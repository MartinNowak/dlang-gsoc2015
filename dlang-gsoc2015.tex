%\documentclass[a4paper,12pt]{article}
%\usepackage[utf8]{inputenc}

%%% LaTeX Template: Two column article
%%%
%%% Source: http://www.howtotex.com/
%%% Feel free to distribute this template, but please keep to referal to http://www.howtotex.com/ here.
%%% Date: February 2011

%%% Preamble
\documentclass[	DIV=calc,%
							paper=a4,%
							fontsize=11pt,%
							twocolumn]{scrartcl}	 					% KOMA-article class

\usepackage{lipsum}													% Package to create dummy text

\usepackage[english]{babel}										% English language/hyphenation
\usepackage[protrusion=true,expansion=true]{microtype}				% Better typography
\usepackage{amsmath,amsfonts,amsthm}					% Math packages
\usepackage[pdftex]{graphicx}									% Enable pdflatex
\usepackage[svgnames]{xcolor}									% Enabling colors by their 'svgnames'
\usepackage[hang, small,labelfont=bf,up,textfont=it,up]{caption}	% Custom captions under/above floats
\usepackage{epstopdf}												% Converts .eps to .pdf
\usepackage{subfig}													% Subfigures
\usepackage{booktabs}												% Nicer tables
\usepackage{fix-cm}													% Custom fontsizes



%%% Custom sectioning (sectsty package)
\usepackage{sectsty}													% Custom sectioning (see below)
\allsectionsfont{%															% Change font of al section commands
	\usefont{OT1}{phv}{b}{n}%										% bch-b-n: CharterBT-Bold font
	}

\sectionfont{%																% Change font of \section command
	\usefont{OT1}{phv}{b}{n}%										% bch-b-n: CharterBT-Bold font
	}



%%% Headers and footers
\usepackage{fancyhdr}												% Needed to define custom headers/footers
	\pagestyle{fancy}														% Enabling the custom headers/footers
\usepackage{lastpage}	

% Header (empty)
\lhead{}
\chead{}
\rhead{}
% Footer (you may change this to your own needs)
\lfoot{\footnotesize \texttt{www.dlang.org} $\cdot$ Layout based on template from \texttt{HowToTeX.com}}
\cfoot{}
\rfoot{\footnotesize page \thepage\ of \pageref{LastPage}}	% "Page 1 of 2"
\renewcommand{\headrulewidth}{0.0pt}
\renewcommand{\footrulewidth}{0.4pt}



%%% Creating an initial of the very first character of the content
\usepackage{lettrine}
\newcommand{\initial}[1]{%
     \lettrine[lines=3,lhang=0.3,nindent=0em]{
     				\color{DarkGoldenrod}
     				{\textsf{#1}}}{}}



%%% Title, author and date metadata
\usepackage{titling}	% For custom titles

\newcommand{\HorRule}{\color{DarkGoldenrod}%% Creating a horizontal rule
\rule{\linewidth}{1pt}%
}

\pretitle{\vspace{-30pt} \begin{flushleft} \HorRule 
				\fontsize{50}{50} \usefont{OT1}{phv}{b}{n} \color{DarkRed} \selectfont 
				}
\title{Digital Mars: 2015 Google Summer of Code Proposal}					% Title of your article goes here
\posttitle{\par\end{flushleft}\vskip 0.5em}

\preauthor{\begin{flushleft}
					\large \lineskip 0.5em \usefont{OT1}{phv}{b}{sl} \color{DarkRed}}
\author{Craig Dillabaugh }											% Author name goes here
\postauthor{\footnotesize \usefont{OT1}{phv}{m}{sl} \color{Black} 
					on behalf of Digital Mars 								% Institution of author
					\par\end{flushleft}\HorRule}

\date{}		

% Title Page
%\title{Digital Mars: 2015 Google Summer of Code Proposal}
%\author{Craig Dillabaugh}


\begin{document}
\maketitle

This document constitutes the organization program proposal for the
2015 Google Summer of Code for Digital Mars, which is the organization
leading development of the D Programming Language. The sections below
generally follow the outline presented in the ``\emph{What should a mentoring
organization proposal look like}'' section of the GSOC FAQ, though some 
elements in the outline were not relevant to our organization and have
been omitted.

\section{Organization}

DigitalMars leads the D Programming Language community.   The D programming
language was initially developed by Walter Bright as a 'better C++', but
has evolved far beyond its original goal.
Recent improvements to 
the D development process have accelerated development of the language and
its standard library, Phobos.
The 
current focus of the D community is to improve the tool support around the 
language, to patch a few remaining holes in the standard library, and to make 
the language more attractive to those currently outside the community.  

Many of the core developers for the D programming language make their 
livelihoods developing with other languages.  D is not yet used extensively
in industry - though it is slowly gaining traction, and has been used very 
successfully in some instances.
D developers generally use it because it makes programming fun for them.  
It is 
powerful, efficient, and doesn't force any particular paradigm on developers,
but rather provides tools that allow them to reason about problems in the way 
they see fit.


\section{Reason For Applying}

In addition to the improvements to the D tool chain and libraries, we 
hope to attract students who will be long term contributors to the D ecosystem.
Many D contributors and users feel that D is very close to gaining more
widespread acceptance, however there are still a few rough edges that
need to be ironed out. 
The Google Summer of Code is attractive to our community because the D 
language is truly a grassroots project with very limited backing from 
large commercial entities.  
Thus, attracting bright and motivated individuals to the community is
is what is needed to move the language ecosystem forward.
The Google Summer of Code is an excellent way to attract such people.


\section{Past Participation}

DigitalMars has submitted proposals to the Google Summer of Code every
summer since 2011 when we first participated. In each of 2011 and 2012
we had 3 student projects accepted, but did not have any successful 
applications in 2013 or 2014.

\section{Past Results}

In 2011 we had three proposals accepted:

\begin{enumerate}
\item \emph{Linear Algebra Library based on SciD} by Cristi Cobzarenco 
 mentored by David Simcha.
\item \emph{An Apache Thrift Implementation for D} by David Nadlinger mentored 
by Nitay Joffe.
\item \emph{Enhance Regular Expressions} by Dmitry Olshansky mentored by
Fawzi Mohamed.
\end{enumerate}

We consider all three projects to have been successes. Cristi rewrote most of the 
original SciD code (by Lars Tandle Kyllingstad).  David completed his project 
and Thrift officially supported D as of March 2012.  Dmitry completed his project 
and his work  now forms one of the modules \texttt{std.regex} in Phobos, the D 
standard library. Both David and Dmitry are still very active in the D community.

In 2012 we also had three projects accepted:

\begin{enumerate}
  \item \emph{Mono-D} by Alex Bothe, mentored by LightBender.
  \item \emph{Removing the global gc lock from common allocations in D} by 
  Antti-Ville Tuuainen, mentored by David Simcha.
  \item \emph{Extended unicode support} by Dimitry Olshansky mentored by
  Andrei Alexandrescu.
\end{enumerate}

We consider that all of these projects were also successful.  Dmitry's extended
unicode support has been accepted into the Phobos standard library as \texttt{std.uni}.
Alex's Mono-D is a popular idea in the D community, is actively developed, and 
Alex is still involved (see \texttt{https://github.com/aBothe/Mono-D}). Antti-Ville's project
turned out to be the most challenging.  The project was redirected to implementing
precise scanning for the GC using RTInfo.  That feature was brand new at the time, so
Antti-Ville and David had to fight with a lot of bugs in DMD (the standard D compiler).
A new implementation of precise garbage collection is being developed by Rainer Schuetze
based in large part on Antti-Ville's work (see 
\texttt{https://github.com/Tuna-Fish/druntime /tree/gc\_poolwise\_bitmap} ).

\section{OSI Licenses}

The various projects making up the D ecosystem use a variety of OSI approved
licenses.  The following table highlights the licenses used by the various projects
for which we have project ideas listed on our projects page.

\begin{center}
   \begin{tabular}{ | l | l | }
     \hline
     \textbf{Project} & \textbf{License} \\ \hline
     DMD & Boost \\
     Phobos & Boost \\
     SDC & MIT \\
     GDC & GPL \\
     DDT & Eclipse Public License/Apache License  \\
     Csmed & MIT \\ \hline
   \end{tabular}
\end{center}


\section{Project Ideas List}

Our project ideas list can be found at:
\begin{itemize}
\item \texttt{http://wiki.dlang.org/GSOC\_2015\_Ideas}
\end{itemize}

\noindent We have also posted a list of bios for potential mentors and their
backups:
\begin{itemize}
\item \texttt{http://wiki.dlang.org/GSOC\_mentors}
\end{itemize}

\section{Main Development Mailing List}
The primary means of communication within the community is
the D Forum:
\begin{itemize}
\item \texttt{http://forum.dlang.org}
\end{itemize}

There are also a number of mailing lists (which are accessible 
through the forum).  A list of the various mailing lists is
available at:

\begin{itemize}
   \item \texttt{http://lists.puremagic.com/mailman/listinfo}
\end{itemize}

\section{IRC channel}
D's main IRC channel is:  \texttt{\#d} on \texttt{irc.freenode.net}

\section{Backup Organization Administrator}
The backup Adminstrator for our organization will be Martin Nowak.

\section{Mentor Selection}

A call was put out on the D community forums for possible mentors.
We asked that the mentors have concrete project ideas that
they were capable of managing development on.  Thus our mentors
have been self selected, but each one has a lengthy involvement 
in the D community. The various proposals were discussed
in the D community forums. 

\section{Disappearing Students}
In order to deal with students who disappear or who fail to
maintain a suitable level of progress in their projects we plan
to do following:

\begin{enumerate}
\item Careful monitoring to avoid such situations, have an agreed upon
schedule.
\item Let the student know the expectations and that we are willing
to fail them if they do not meet expectations - if they are not prepared 
to take action to correct the situation.
\item Prepare to have community take over project in the worst case.
\end{enumerate}

\section{Disappearing Mentors}

We plan to identify backup mentors for all accepted projects,
and keep in close contact with our mentors throughout the
summer.  Not only will students be expected to report to the mentors,
but the administrator will follow up regularly with the mentors, and
keep communication channels open with the students.

We have a number of volunteer backups for the mentors, in case
some mentor becomes unable to fulfill their duties for whatever
reason.  Our backup mentors include:

\begin{enumerate}
     \item \textbf{Adam D. Ruppe}  Adam has many years of experience in
     D development and recently released a book on D programming, \\
     'D Cookbook'.
     \item \textbf{Dmitry Olshansky}  Twice a successful participant
     in the Google Summer of Code as a student, Dmitry continues to be 
     very active in the D community. 
\end{enumerate}

\section{Community Interaction}
\emph{What steps will you take to encourage students to interact 
with your project's community before, during and after 
the program?}

Some steps we plan to take to help students interact with
the D community:

\begin{enumerate}
\item Have students post regular (monthly) progress reports
on the D forum, to keep the community up to date, but also
to attract feedback and solicit ideas.
\item The annual D conference, DConf, is being held at Utah
Valley State University in Orem, Utah from May 27\textsuperscript{th}
to May 29\textsuperscript{th}, 2015. This is conveniently right at 
the end of the student bonding period, and we will encourage any
GSOC students to attend this meeting where it is financially 
feasible. See (\texttt{http://dconf.org/2015/index.html}) for information on 
the 2015 edition of DConf.
\end{enumerate}


\section{Encouraging Students to Stay Involved After GSOC}


When evaluating student project proposals we will try to identify
students and projects that are good candidates for continued work
after the project is completed.  If student's experience with the 
D community is positive, and their GSOC project is successful, then
we feel there is a good chance that they will continue with their
involvement in project. We have had very good experience with retaining
students during past participation in the Goolge Summer of Code.     



\end{document}
