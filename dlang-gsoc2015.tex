\documentclass[a4paper,12pt]{article}
\usepackage[utf8]{inputenc}

% Title Page
\title{Digital Mars: 2015 Google Summer of Code Proposal}
\author{Craig Dillabaugh}


\begin{document}
\maketitle

\section{Organization}

DigitalMars leads the D Programming Language community.   The D programming
langauge was initially developed by Walter Bright, as a 'better C++', but
has evolved far beyond its original goal.
Recent improvements to 
the D development process have accelarated development of the language. The 
current focus of the D community is to improve the tool support around the 
language to make the language more attractive to those currently outside the 
community.  

Many of the core developers for the D programming language make thier 
livelihoods developing with other languages.  D is not, yet, used extensively
in industry - though it is slowly gaining traction, and has been used very 
successfully in some instances.
D developers generally use it because it makes programming fun again for them.  
It is 
powerful, efficient, and doesn't force any particular paradigm on developers,
but rather provides tools that allow them to reason about problems in the way 
they see fit. 

D developers, and users, feel that D is already an excellent language, but 
that in order
for it to gain wider acceptance it needs better tools, and some additional 
work on its standard library (Phobos).  


\section{Reason For Applying}

In addition to the improvements to the D toolchain and libraries, we 
hope to attract students who will be long term contributors to the D ecosystem.
Many D contributors and users feel that D is very close to gaining more
widespread acceptance, however there are still a few rough edges that
need to be ironed out. 
The Google Summer of Code is attractive to our community because the D 
language is truly a grassroots project with very limited backing from 
large commercial entities.  
Thus, attracting bright, and motivated, individuals to the community is
is what is needed to move the langauge ecosystem forward.
The Google Summer of Code is an excellent way to attract such people.



\section{Past Participation}

DigitalMars has submitted proposals to the Google Summer of Code every
summer since 2011 when we first participated. In each of 2011 and 2012
we had 3 student projects accepted, but did not have any successful 
applications in 2013 or 2014.

\section{Past Results}

In 2011 we had three proposals accepted:

\begin{enumerate}
\item \emph{Linear Algebra Library based on SciD} by Cristi Cobzarenco 
 mentored by David Simcha.
\item \emph{An Apache Thrift Implementation for D} by David Nadlinger mentored 
by Nitay Joffe.
\item \emph{Enhance Regular Expressions} by Dmitry Olshansky mentored by
Fawzi Mohamed.
\end{enumerate}

We consider all three projects to have been successes. Cristi rewrote most of the 
original SciD code (by Lars Tandle Kyllingstad).  David completed his project 
and Thrift officially supported D as of March 2012.  Dmitry completed his project 
and his work  now forms one of the modules \texttt{std.regex} in Phobos, the D 
standard library. Both David and Dmitry are still very active in the D community.

In 2012 we also had three projects accepted:

\begin{enumerate}
  \item \emph{Mono-D} by Alex Bothe, mentored by LightBender.
  \item \emph{Removing the global gc lock from common allocations in D} by 
  Antti-Ville Tuuainen, mentored by David Simcha.
  \item \emph{Extended unicode support} by Dimitry Olshansky mentored by
  Andrei Alexandrescu.
\end{enumerate}

We consider that all of these projects were also successful.  Dmitry's extended
unicode support has been accepted into the Phobos standard library as \texttt{std.uni}.
Alex's Mono-D is a popular idea in the D community, is actively developed, and 
Alex is still involved (see https://github.com/aBothe/Mono-D). Antti-Ville's project
turned out to be the most challenging.  The project was redirected to implementing
precise scanning for the GC using RTInfo.  That feature was brand new at the time, so
Antti-Ville and David had to fight with a lot of bugs in DMD (the standard D compiler).
A new implementation of precise garbage collection is being developed by Rainer Schuetze
based in large part on Antti-Ville's work (see 
https://github.com/Tuna-Fish/druntime/tree/gc\_poolwise\_bitmap ).

\section{Placeholder}

If your organization has not previously participated in Google Summer of Code, 
have you applied in the past? If so, for what year(s)?

\section{OSI Licenses}

What Open Source Initiative approved license(s) does your project use?


\section{Project Ideas List}

\texttt{http://wiki.dlang.org/GSOC\_2015\_Ideas}

\section{Main Development Mailing List}
What is the main development mailing list for your organization?

\section{IRC channel}
What is the main IRC channel for your organization?

\section{Backup Organization Administrator}
Who will be your backup organization administrator?

\section{Mentor Selection}

What criteria did you use to select the mentors? 
Please be as specific as possible.

A call was put out on the D community forums for possible mentors.
We asked that the mentors have concrete project ideas that
they were capable of managing development on.  Thus our mentors
have been self selected, but each one has a lengthy involvement 
in the D community. The various proposals were discussed
in the D community forums. 

\section{Disappearing Students}
What is your plan for dealing with disappearing students? 
Please be as specific as possible.

911

\section{Disappearing Mentors}

We plan to identify backup mentor's for all accepted projects, 
and keep in close contact with our mentor's throughout the
summer.  

\section{Community Interaction}
What steps will you take to encourage students to interact 
with your project's community before, during and after 
the program?

Some steps we plan to take to help students interact with
the D community:

\begin{enumerate}
\item Have students post regular (monthly) progress reports
on the D forum, to keep the community up to date, but also
to attract feedback and solicit ideas.
\item 
\end{enumerate}

\section{Googlers}
Are you a new organization who has a Googler or other organization
to vouch for you? If so, please list their name(s) here.

Not applicable.

\section{Who can we vouch for}
Are you an established or larger organization who would like to 
vouch for a new organization applying this year? If so, please 
list their name(s) here.

Vibe D.   Anyone?

\section{Stick to It}
What will you do to encourage that your accepted students stick 
with the project after Google Summer of Code concludes?

We have tried to present interesting projects that will be fun
to work on.




\end{document} 